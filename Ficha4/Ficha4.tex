\documentclass[a4paper, 11pt]{report}

\usepackage[portuguese]{babel}
\usepackage{multirow}

\title{Aula de Dúvidas de Produção de Relátorios}
\author{Arthur Aguiar}
\date{\today}

\begin{document}

\maketitle

\chapter{}

\section{Tabela sem linhas}

\begin{table}[h!]
	\centering
	\begin{tabular}{l c r}
		1      & 2        & 3      \\
		4      & 5        & 6      \\
		grande & enorme   & gigante \\
	\end{tabular}
\end{table}

\section{Tabela com uma linha}

\begin{table}[h!]
	\centering
	\begin{tabular}{ l | c r }
		1      & 2        & 3      \\
		4      & 5        & 6      \\
		grande & enorme   & gigante \\
	\end{tabular}
\end{table}


\section{Tabela com duas linhas}

\begin{table}[h!]
	\centering
	\begin{tabular}{ l | c | r}
		1      & 2        & 3      \\
		4      & 5        & 6      \\
		grande & enorme   & gigante \\
	\end{tabular}
\end{table}



%  Tres primeiras tabelas

\section{Tabela com duas linhas a esquerda}

\begin{table}[h!]
	\centering
	\begin{tabular}{ |l | c  r}
		1      & 2        & 3      \\
		4      & 5        & 6      \\
		grande & enorme   & gigante \\
	\end{tabular}
\end{table}

\section{Tabela com quatro linhas}
\begin{table}[h!]
	\centering
	\begin{tabular}{ | l | c|  r|}
		1      & 2        & 3      \\
		4      & 5        & 6      \\
		grande & enorme   & gigante \\
	\end{tabular}
\end{table}

\section{Tabela com quatro linhas verticais e uma Horizontal}

\begin{table}[h!]
	\centering
	\begin{tabular}{ | l | c|  r|}
		\hline
		1      & 2        & 3      \\
		4      & 5        & 6      \\
		grande & enorme   & gigante \\
	\end{tabular}
\end{table}

\section{Tabela com duas linhas verticais e duas linhas horizontais}

\begin{table}[h!]
	\centering
	\begin{tabular}{ | l | c|  r|}
		\hline
		1      & 2        & 3      \\
		4      & 5        & 6      \\
		\hline
		grande & enorme   & gigante \\
	\end{tabular}
\end{table}


\section{Tabela com duas linhas verticais e duas linhas horizontais}

\begin{table}[h!]
	\centering
	\begin{tabular}{ | c| c|  c|}
		\hline
		1      & 2        & 3      \\
		4      & 5        & 6      \\
		\hline
		grande & enorme   & gigante \\
	\end{tabular}
\end{table}

\section{Tabela com duas linhas verticais e duas linhas horizontais centrada}

\begin{table}[h!]
	\centering
	\begin{tabular}{ | c | c |  c|}
		\hline
		1      & 2        & 3      \\
		\hline
		4      & 5        & 6      \\
		\hline
		grande & enorme   & gigante \\
	\end{tabular}
\end{table}

\section{Tabela com duas linhas verticais e tres linhas horizontais centrada}

\begin{table}[h!]
	\centering
	\begin{tabular}{ | c | c |  c|}
		\hline
		1      & 2        & 3      \\
		\hline
		4      & 5        & 6      \\
		\hline
		grande & enorme   & gigante \\
		\hline
	\end{tabular}
\end{table}

\section{Legenda em cima}

\begin{table}[h!]
	\centering
	\caption{legenda}
	\begin{tabular}{ | c | c |  c|}
		\hline
		1      & 2        & 3      \\
		\hline
		4      & 5        & 6      \\
		\hline
		grande & enorme   & gigante \\
		\hline
	\end{tabular}
\end{table}

\section{Legenda em baixo}

\begin{table}[h!]
	\centering
	\begin{tabular}{ | c | c |  c|}
		\hline
		1      & 2        & 3      \\
		\hline
		4      & 5        & 6      \\
		\hline
		grande & enorme   & gigante \\
		\hline
	\end{tabular}
	\caption{legenda}
\end{table}


\section{Colunas unidas}

\begin{table}[h!]
	\centering
	\begin{tabular}{ | c | c |  c|}
		\hline
		\multicolumn{3}{ | c | }{Coluna unida} \\
		\hline
		1      & 2        & 3      \\
		\hline
		4      & 5        & 6      \\
		\hline
		grande & enorme   & gigante \\
		\hline
	\end{tabular}
	\caption{legenda}
\end{table}

\newpage

\section{Linhas unidas}
\begin{table}[h!]
	\centering
	\begin{tabular}{ |c | c | c |  c|}
		\hline
		\multirow{3}{*}{Linhas Unidas}
		& 1      & 2        & 3      \\
		& 4      & 5        & 6      \\
		& grande & enorme   & gigante \\ \hline
	\end{tabular}
\end{table}

\end{document}