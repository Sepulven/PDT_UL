\documentclass[a4paper,12pt]{article}

\usepackage[portuguese]{babel}
\usepackage[utf8]{inputenc}
\usepackage[normalem]{ulem}
\usepackage{xcolor}


\title{A vida do Suricata}
\author{Arthur Aguiar}
\date{\today}
\begin{document}
\maketitle

\section{Sobre o Suricata}
O \textbf{suricata}, também chamado de  \textbf{suricato} ou  \textbf{suricate} \textit{(\textbf{Suricata} su-
ricatta)} é um pequeno mamífero da família \textit{Herpestidae}, nativo do
deserto do Kalahari. Estes animais têm cerca de meio metro de com-
primento (incluindo a cauda), em média 730 gramas de peso, e pela-
gem acastanhada. Têrm garras afiadas nas patas, que lhes permitem
escavar a superfície do chão e dentes afiados para penetrar nas ca-
rapaças quitinosas das suas presas. \textcolor{red}{Outra característica distinta é a sua capacidade de se elevarem nas patas traseiras, utilizando a cauda como terceiro apoio.}

\section{Características gerais}
\subsection{Alimentação}
Alimenta-se principalmente de insetos (cerca de 82 \%): lar-vas de escaravelhos e de borboletas; também ingerem milíedes, aranhas, escorpi˜ões, pequenos vertebrados (répteis, anfíbios e aves), ovos
e matéria vegetal. \uline{São relativamente imunes ao veneno} das najas
e dos escorpiões, sendo estes, inclusive, um dos alimentos que mais
apreciam.

\end{document}
