\documentclass[a4paper, 11pt]{article}
\usepackage[portuguese]{babel}

% temos de utilizar o pacote do enumerate, para conseguir criar uma lista padronizada
\usepackage{enumerate}

\usepackage{amsmath}
\usepackage{amsfonts}
\usepackage{amssymb}

\begin{document}

\section{Exércicios}

\begin{enumerate}[a.]
	\item
	\[ 
		\begin{pmatrix}
			a \\
			b + c
		\end{pmatrix}
		\begin{pmatrix}
			\frac{n^{2}-1}{2} \\
			n + 1
		\end{pmatrix}
	\]

	\item
	$
		\begin{pmatrix}
			n \\
			k
		\end{pmatrix}
		=
		\frac{n!}{k!(n - k!)}
	$

	\item \[
		A = \begin{bmatrix}
				1 & 3 & 0 \\
				2 & 4 & -2
			\end{bmatrix}
		\]	

	\item $
			B = \begin{Bmatrix}
				1 & 3 & -2
			\end{Bmatrix}
		  $

	\item \[ 
			\mathrm{C}=\begin{pmatrix}
						1 \\
						3 \\
						-4
					\end{pmatrix}
		  \]


	\item $
			A = \begin{pmatrix}
				a & b & c \\
				d & e & f \\
				g & h & i
			\end{pmatrix}
		$

	\item \[ 
			\begin{Bmatrix}
				a & b & c \\
				d & e & f \\
				g & h & i
			\end{Bmatrix}
		  \]

	\item $
		\begin{vmatrix} A \end{vmatrix} = \begin{vmatrix}
											1 & 2 & 3 \\
											4 & 5 & 6 \\
											7 & 8 & z
										  \end{vmatrix}
		 $
	
	\item \[ 
		\begin{Vmatrix} A \end{Vmatrix} = \begin{Vmatrix}
											t & n & m \\
											b & d & o \\
											b & s & x
										\end{Vmatrix}
			\]

	\item $ 
		A = \begin{Bmatrix}
			\vdots & \ldots & \ddots\\
		\end{Bmatrix} 
		$
	% \item \begin{displaymath}
	% 		\mathbf{A}=\left(\begin{array}{cc}
	% 		a & b\\
	% 		c & d
	% 		\end{array}\right)
	% 	 \end{displaymath}
\end{enumerate}

\end{document}
