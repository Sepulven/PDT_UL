\documentclass[a4paper, 11pt]{article}
\usepackage[portuguese]{babel}
% O pacote de geometria é utilizado para fazer o controlo das margens.
\usepackage{geometry}
% \usepackage[right=1cm, top=1cm, bottom=1cm, left=1cm]{geometry}

% hyper links -> hyperref latex
% Exercicio para treinar para o teste.


\begin{document}
	\section{A Descoberta dos seres Peixes}
		{\hskip 3 mm }
		\textit{Seres primitivos, dotados de armadura, de placas espinhosas e escamas sobrepostas,
		sem mandíbula e com três olhos na cabeça}
		deram origem, há
		{\LARGE 400 milhões}
		de anos, às 30.000 espécies de peixes que conhecemos hoje.

		\vspace{0.5cm}
		% \sc poem os carecteres com letras minusculo com aparência de letra maiuscula
		{\huge \sc{Como contar a sua idade?}} 
		\vspace{0.5cm}


		{\hskip 3mm} As escamas apresentam-se com os aspetos mais variados, mas se examinarmos
		qualquer delas com atenção, veremos uma série de círculos; através deles podemos saber a
		idade dos peixes. Mas, a cada círculo não corresponde um ano de crescimento.

		\textbf {Um peixe não cresce sempre o mesmo através do ano.} Nos meses
		{\tt quentes}, quantidade de anéis bastante distanciados. Nos
		\tt{meses frios} , os peixes crescem muito devagar e ás vezes nem chegam a crescer.
		Os seus anéis de crescimento são poucos e muitos fundos.

		Este conjunto de anéis forma uma espécie de linha mais acentuada. Para
		conhecermos com exatidão a idade de um peixe 
		{\emph devemos contar apenas linhas}
		pois, uma, marca a passagem de um {\huge inverno}.
	\section{Valor Nutricional do Peixe}


\end{document}