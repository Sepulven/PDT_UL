\documentclass[a4paper,11pt]{report}

\usepackage[portuguese]{babel}
\usepackage{graphicx}
\usepackage{titlepic}


% 1. Note:

% In order for the label and reference to work we need to compile the .tex
% file twice.

% 2. Note:

% In order for the right positioning to work with table we must add the 
% [htbp] property.
\author{Arthur Aguiar}
\title{A série TV Guerra dos Tronos}
\titlepic{
	\includegraphics[width=12cm]{./game-of-thrones.jpg}
}

\begin{document}

%Display the title in the first page
\maketitle

	\chapter{Introdução}
	\section{Sobre a Série}
	\label{intro}

		A {\Huge Guerra dos Tronos}\footnote{Este nome é uma marca registada.} 
		é uma superprodução Televisiva da HBO,
		baseada na saga literária de George R.R. Martin, é uma série
		que {\tiny redefine os parâmetros do que é possível fazer em televisão.}

		\vspace{1cm} % Vertical spacing

		\textit{Uma narrativa épica que} {\huge atravessa mundos imaginários}

		\vspace{-0.5cm} % We need to fix the spacing that is affected by the the previous vspace

		\begin{center}
			e personagens a perder de vista
			\footnote{Texto retirado de http://tv.sapo.pt/series/a-guerra-dos-tronos.}.
		\end{center}

		\vspace{1cm} % Vertical spacing

		% {} empty braces encloses empty groups, ^ and ~ are treated 
		{\tt UPS!Este texto não devia estar aqui / \S \" \% \$ \& \^{} \~{}}

		\section{Elenco}
		\label{elenco}

		A secção \ref{intro} falou-nos sobre a série e agora na secção \ref{elenco} vamos falar sobre
		o elenco. A tabela \ref{intro} mostra-nos as personagens principais.

\begin{table}[h!] % htbp helps latex with placing the table.
	\centering
	\begin{tabular}{ | l | c | r | } % Last column alignment is to the left
		\hline % Top border
		\textbf{Ator/Atriz} & \textbf{Personagem}  & \textbf{Temporada} \\
		\hline %Horizontal border
		Peter Dinklage      & Tyrion Lannister   & 1, 2 e 3  \\
		\hline %Horizontal border
		Emilia Clarke       & Daenerys Targaryen & 1, 2 e 3  \\
		\hline %Horizontal border
		Richard Madden      & Robb Stark         & 1 e 2     \\
		\hline %Bottom border
	\end{tabular}

	\caption{Principais personagens da “Guerra dos Tronos” ao longo das
	várias temporadas.}
\end{table}

\end{document}