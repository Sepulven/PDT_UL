\documentclass[a4paper,11pt]{report}

\usepackage[portuguese]{babel}
\usepackage[utf8]{inputenc}
\usepackage[normalem]{ulem}
\usepackage{xcolor}
\usepackage{colortbl} %So we can paint the background of the first row

\title{A vida do Suricata}
\author{Arthur Aguiar}
\date{\today}
\begin{document}
\maketitle

\chapter{Introdução}

\section{Sobre o Suricata}
	O \textbf{suricata}, também chamado de \textbf{suricato} ou \
	\textbf{suricate} \textit{(\textbf{Suricata} su-\
	ricatta)} é um pequeno mamífero da família \textit{Herpestidae}, nativo do\
	deserto do Kalahari. Estes animais têm cerca de meio metro de com-\
	primento (incluindo a cauda), em média 730 gramas de peso, e pela-\
	gem acastanhada. Têrm garras afiadas nas patas, que lhes permitem
	escavar a superfície do chão e dentes afiados para penetrar nas ca-
	rapaças quitinosas das suas presas. \textcolor{red}{Outra característica distinta é a sua capacidade de se elevarem nas patas traseiras, utilizando a cauda como terceiro apoio.}

\section{Características gerais}
\subsection{Alimentação}
	Alimenta-se principalmente de insetos (cerca de 82 \%):

	\begin{itemize}
		\item lar-vas de escaravelhos e de borboletas;
		\item milíedes;
		\item aranhas;
	\end{itemize}

..mas também de

	\begin{itemize}
		\item escorpiões;
		\item pequenos vertebrados (répteis, anfíbios e aves);
		\item ovos;
		\item matéria vegetal.
	\end{itemize}

	\uline{São relativamente imunes ao veneno} das najas
	e dos escorpiões, sendo estes, inclusive, um dos alimentos que mais
	apreciam.

\chapter{Desenvolvimento}

\section{Onde avistar suricatas no habitat selvagem?}
	Existem vários parques nacionais em Àfrica onde è possìvel avistar e
	atè interagir com suricatas no seu habitat selvagem. No entanto, existe
	uma regra de ouro: os suricatas não gostam de chuva, por isso prefira
	dias solarengos.
	Em baixo apresenta-se uma lista de parques ordenada por número de
	suricatas por Km2:

	\begin{enumerate}
		\item Parque “Kgalagadi”,\textit{África do Sul / Botswana}
		\item Parque nacional “Karoo”, \textit{África do Sul}
		\item Reserva do vale mágico do Suricata, \textit{África do Sul}
		\item Parque nacional Iona, \textit{Angola}
	\end{enumerate}

ou

	\begin{description}
		\item[parque1] Parque “Kgalagadi”, \textit{África do Sul / Botswana}
		\item[parque2] Parque nacional “Karoo”, \textit{África do Sul}
		\item[parque3] Reserva do vale mágico do Suricata, \textit{África do Sul}
		\item[parque4] Parque nacional Iona, \textit{Angola}
	\end{description}


\section{Subspécie}
	\textit{Existem atualmente três subespécies de Suricata:}


	\begin{itemize}
		\item \textit{Suricata suricatta siricata;}
		\item \textit{Suricata suricatta iona;}
		\item \textit{Suricata suricatta majoriae.}
	\end{itemize}

	\textit{Os indivíduos de cada subspécie apresentam características \
	distintas como se pode ver na tabela e na figura seguintes.}

	\begin{table}[h!]
		\begin{tabular}{|c|l|l|l|}
			\hline 
			\rowcolor{gray!40}
			\multicolumn{1}{|l|}{} &
			\multicolumn{1}{c|}{\textbf{siricata}} &
			\multicolumn{1}{c|}{\textbf{iona}} &
			\multicolumn{1}{c|}{\textbf{majoriae}} \\ \hline
			\textbf{côr do pelo} & beje amarelado & castanho amarelado & castanho escuro \\ \hline
			\textbf{tamanho}     & 29 cm          & 25 cm              & 34 cm           \\ \hline
			\textbf{peso}        & 731g           & 698g               & 799g            \\ \hline
		\end{tabular}
		\caption{Tabela com a primeira linha colorida}
	\end{table}

\end{document}
