\documentclass[a4paper, 11pt]{article}
\usepackage[portuguese]{babel}

% temos de utilizar o pacote do enumerate, para conseguir criar uma lista padronizada
\usepackage{enumerate}

\usepackage{amsmath}
\usepackage{amsfonts}
\usepackage{amssymb}

\begin{document}

\section{Símbolos Matemáticos}

\begin{enumerate}[A)]
	\item Se $L(x)$ representar "x tem cabelo louro", então a frase (1) escreve-se
	simbolicamente $\forall, L(x)$. A sua negação, que poderíamos coloquialmente redigir
	"nem todas as pessoas têm cabelo louro", escreve-se $\neg\forall, L(x)$ e é logicamente equivalente
	a (2), que se escreve $\exists x, \neg L(x)$.

	\item Se $A \subseteq B$ e $B \subseteq A$, então $A$ e $B$ têm extamente os mesmos elementos e, portanto,  
		$A = B$.
	\item Sejam $A = 2, 3, 5, 7, 9$, $B = 1, 3, 5, 7, 9$, $C = N$
		  $\{D = x \in  \mathbb{Z}  \mid x < 5 \}$
	\item  $\lim_{x \to \inf} \exp(-x) = 0$
	\item Em trigonometria, a relação básica entre o seno e o coseno é conhecida 
	como \emph{Identidade Trigonométrica Fundamental}: $ \cos ^2(\theta) + \sin ^2(\theta) = 1$ 
\end{enumerate}


\section{Equações}	

\begin{enumerate}[i)]
	\item Uma equação com duas linhas e alinhada pelo sinal de igualdade:
		\begin{equation}
			\begin{split}
				x^2 + z ^3 & = \sqrt{2+3y} \\
				  x \div 5 & = z^{x+2\pi}
			\end{split}
		\end{equation}

	\item A equação anterior dividida em duas equações alinhadas pelo sinal de igualdade:
			\begin{align}
				x^2 + z ^3 & = \sqrt{2+3y} \\
				x \div 5 & = z^{x+2\pi}
			\end{align}
	
	\item Três equações com três colunas (repare com atenção nos alinhamentos):
		\begin{align}
			X_{a}  & =\sqrt{x+y} & X_{b}  & = \pi + 3                 & x_{c}v & = 2 + x^{y} \\
			Z_{ax} & =x^{3} + 7  & Z_{ay} & = \sqrt{x^{4} + 5Y} + 27y & Y_{a}  & = x - 2 - y \\
			Z_{ax} & = 10        & Z_{ay} & = 35y - 2                 & Z_{az} & = 2 + y
		\end{align}

	\item Represente o seguinte sistema:
	\begin{equation}
		f(x)=
		\begin{cases}
			1 & \text{se } x \geq 5 \text{,} \\
			0 & \text{se } 1 < x < 2 \text{,} \\
			-1 & \text{se } x \leq  1 \text{.}
		\end{cases}
	\end{equation}
\end{enumerate}

\end{document}